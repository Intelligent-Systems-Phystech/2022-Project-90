\documentclass{article}
\usepackage{arxiv}

\usepackage[utf8]{inputenc}
\usepackage[english, russian]{babel}
\usepackage[T1]{fontenc}
\usepackage{url}
\usepackage{booktabs}
\usepackage{amsfonts}
\usepackage{nicefrac}
\usepackage{microtype}
\usepackage{lipsum}
\usepackage{graphicx}
\usepackage{natbib}
\usepackage{doi}


\renewcommand{\shorttitle}{\textit{arXiv} Template}
\renewcommand{\abstractname}{Аннотация}

\title{Восстановление движения руки по видео}

\author{Владимиров Эдуард \\
	\texttt{vladimirov.ea@phystech.edu} \\

	\And
	Исаченко Роман \\
	\texttt{isa-ro@yandex.ru} \\
	
	\And
	Курдюкова Антонина \\
	\texttt{kurdiukova.ad@phystech.edu} \\
	%% \And
	%% Coauthor \\
	%% Affiliation \\
	%% Address \\
	%% \texttt{email} \\
	%% \And
	%% Coauthor \\
	%% Affiliation \\
	%% Address \\
	%% \texttt{email} \\
}
\date{\today}

%%% Add PDF metadata to help others organize their library
%%% Once the PDF is generated, you can check the metadata with
%%% $ pdfinfo template.pdf
% \hypersetup{
% pdftitle={A template for the arxiv style},
% pdfsubject={q-bio.NC, q-bio.QM},
% pdfauthor={David S.~Hippocampus, Elias D.~Striatum},
% pdfkeywords={First keyword, Second keyword, More},
% }

\begin{document}
\maketitle

\begin{abstract}
	Ставится задача предсказания временного ряда со сложной структурой, то есть наличием различных зависимостей и 
	варьирующейся периодичности. Поскольку исходная и целевая переменные связаны между собой, то 
	для обнаружения этой связи предлагается использовать метод сходящихся перекрёстных отображений. Решение 
	данной задачи продемонстрировано на фрагменте видео движущегося человека и показаниях акселерометра.
\end{abstract}


\keywords{Временные ряды \and Фазовые траектории \and More}

\section{Введение}

\section{Headings: first level}

\subsection{Headings: second level}

\subsubsection{Headings: third level}

\paragraph{Paragraph}


\section{Examples of citations, figures, tables, references}

\subsection{Citations}

\subsection{Figures}


\subsection{Tables}

\subsection{Lists}


\bibliographystyle{unsrtnat}
\bibliography{references}

\end{document}