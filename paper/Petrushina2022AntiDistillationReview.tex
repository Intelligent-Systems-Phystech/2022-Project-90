\documentclass[a4paper, 12pt]{article}
\usepackage[T2A]{fontenc}
\usepackage[utf8]{inputenc}
\usepackage[english, russian]{babel}
% \usepackage{cmap}
\usepackage{nicefrac}
\usepackage{microtype}
\usepackage{lipsum}

%% Шрифты
\usepackage{euscript} % Шрифт Евклид
\usepackage{mathrsfs} % Красивый матшрифт
\usepackage{extsizes} % Возможность сделать 14-й шрифт

\usepackage{amsmath,amsfonts,amssymb,amsthm,mathtools,dsfont}
\usepackage{icomma}

\usepackage{hyperref}
\usepackage[usenames,dvipsnames,svgnames,table,rgb]{xcolor}

\hypersetup{
	unicode=true,
	pdftitle={A template for the arxiv style},
	pdfsubject={q-bio.NC, q-bio.QM},
	pdfauthor={David S.~Hippocampus, Elias D.~Striatum},
	pdfkeywords={First keyword, Second keyword, More},
	colorlinks=true,
	linkcolor=black,        % внутренние ссылки
	citecolor=blue,         % на библиографию
	filecolor=magenta,      % на файлы
	urlcolor=blue           % на URL
}

\begin{document}
	\begin{center}
		Рецензия на работу
		
		Антидистилляция: передача знаний от простой модели к сложной
		
		Петрушина Ксения
	\end{center}
В работе рассматривается техника антидистилляции, которая позволяет увеличить диверсификацию обучаемой модели. 
Для достижения этой цели используются различные регуляризационные функции потерь. 
Новизна работы заключается в использовании различных техник инициализации весов модели-ученика. 

\begin{enumerate}
	\item Во введении было бы здорово ввести мотивацию для введения антидистилляции. Например, как \href{https://arxiv.org/pdf/2010.09923.pdf}{здесь}
	\item Стоит дать определение для функции $\varphi$ из постановки задачи
	\item Необходимо указать в вычислительном эксперименте, какой оптимизатор используется при обучении моделей.
	\item Не совсем понятно, зачем использовать инициализация нулями, ведь таким образом слой нейросети вырождается в один нейрон. \href{https://medium.com/@safrin1128/weight-initialization-in-neural-network-inspired-by-andrew-ng-e0066dc4a566#:~:text=Zero%20initialization%3A&text=If%20all%20the%20weights%20are,will%20produce%20a%20poor%20result.}{ссылка}
	\item Стоит добавить ещё несколько способов инициализации весов (например, через нормальное распределение), а также несколько метрик качеств для будущей таблицы.
	\item В теоретической части фигурирует некая функция $\psi$, определение которой мне найти не удалось.
\end{enumerate}

Текст статьи с пометками на перечисленные выше и другие замечания направлен автору.  

\textbf{Итог}

Работа хорошая и рекомендуется к публикации. 
Качественно описаны постановка задачи и детали эксперимента, по ходу чтения статьи прослеживается основной посыл работы. 
Но стоит доработать некоторые недочеты по тексту статьи.

Рецензент:

Владимиров Э.А.
\end{document}